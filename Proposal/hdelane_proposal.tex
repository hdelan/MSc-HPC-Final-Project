\documentclass[a4paper, fleqn]{article}

\date{\today}
\author{Hugh Delaney}
\title{MSc in HPC Project Proposal}

\usepackage[utf8]{inputenc}
\usepackage[T1]{fontenc}
\usepackage{textcomp}
\usepackage{amsmath, amssymb, amsthm}
\usepackage{mathtools, geometry}
\geometry{left=2.5cm,right=2.5cm,top=2.5cm,bottom=2.5cm}
\usepackage{xcolor}
\usepackage{listings}

\DeclarePairedDelimiterX{\inp}[2]{\langle}{\rangle}{#1, #2}

\setlength{\mathindent}{1cm}

% figure support
\usepackage{import}
\usepackage{xifthen}
\pdfminorversion=7
\usepackage{subfigure, pdfpages}
\usepackage{transparent}
\newcommand{\incfig}[1]{%
        \def\svgwidth{\columnwidth}
        \import{./figures/}{#1.pdf_tex}
}

\pdfsuppresswarningpagegroup=1

\newtheorem{theorem}{Theorem}[section]
\newtheorem{definition}[theorem]{Definition}
\newtheorem{corollary}{Corollary}[theorem]
\newtheorem{proposition}{Proposition}[theorem]
\newtheorem{lemma}[theorem]{Lemma}
\newtheorem{remark}{Lemma}[theorem]

\renewcommand\qedsymbol{$\blacksquare$}

\begin{document}
\maketitle
\begin{tabbing}
        \textbf{Project Supervisor:}~~~~~~~ \=Kirk Soodhalter \\
        \textbf{Secondary Supervisor:} \>Jose Refojo \\
        \textbf{Research Area:}  \> Computational Graph Theory \\
\end{tabbing}

\section*{Project Brief}%
\label{sec:project_brief}
Compute the action of exp(A) on a vector, where A is an  adjacency matrix of an undirected graph. The graph exponential is generally computed using Krylov methods, most notably using the Lanczos algorithm with some preconditioning. Parallel Lanczos methods to compute the exponential are not common in academic literature, which may be a challenge. See (1) and (2) 

\section*{Preliminaries}%
\label{sec:preliminaries}

Define a symmetric diagonally dominant matrix (SDD) matrix as a matrix where 
\[  |a_{ii}| \ge \sum_{i\neq j} a_{ij} \]

\section*{Implementation Details}%
\label{sec:project_brief}

I will aim to compute the action of exp(A) on a vector using a parallel Lanczos-based algorithm. The implementation will ideally use both MPI and CUDA.
\\ \\ A brief outline of steps in the project:

\subsection*{Primary Goals}%
\label{sub:primary_goals}

\begin{itemize}
        \item Writing a parallel Lanczos algorithm to compute exp(A) on a sparse Symmetric Diagonally Dominant (SDD) graph, using the METIS graph partitioning library to split the graph among processes.
                \begin{itemize}
                        \item In parallel using MPI
                        \item In parallel using MPI and CUDA
                \end{itemize}
        \item Modifying the code to accomodate non SDD matrices increasingly irregular graphs, with partitions being those outputted by METIS. The levels of complexity involved are as followed:
                \begin{itemize}
                        \item SDD graphs which can be neatly partitioned among processes without overlapping edges between processes (this is equivalent to a master-slave serial algorithm).
                        \item SDD graphs with 
                \end{itemize}
\end{itemize}

\subsection*{Secondary Goals}%
\label{sub:subsection_name}


\begin{itemize}
        \item Writing a serial graph-partitioning algorithm in the same vein as the METIS graph partitioning library, which is suitable for scale-free graphs, or graphs whose nodes' degrees follow a power law distribution, making the graphs highly irregular.
        \item Making the Lanczos algorithm run using the graph partitions found from self-written graph partitioning algorithm. Comparing the results with the results from METIS partition.
\end{itemize}

\subsection*{Partial Completion}%
\label{sub:partial_completion}

The project will aim to complete all of the Primary Goals, and will only go on to the Secondary Goals once all Primary Goals have been accomplished. In the event that the Primary Goals take longer than anticipated, the project will terminate at a reasonable date, with as many Secondary Goals completed as possible (potentially none). The order in which Secondary goals are attacked is liable to change from the sequential order listed here.

\section*{References}%
\label{sec:references}
\begin{enumerate}
        \item Jasper vanden Eshof and Marlis Hochbruck. \textit{Preconditioning lanczos approximations to the matrix exponential.} SIAM J. Sci. Comput., 27:1438–1457, November 2005.
        \item Lorenzo Orecchia, Sushant Sachdeva, Nisheeth K. Vishnoi. \textit{Approximating the Exponential, the Lanczos Method and an O(m)-Time Spectral Algorithm for Balanced Separator}
\end{enumerate}


\end{document}
